
Beban tenaga kesehatan sangat tinggi dalam menangani pasien \covid. Robot \covid\ dikembangkan untuk memberi bantuan tenaga dalam perawatan pasien, pengawasan, pelayanan dan membantu mobilitas. Sistem pendeteksian manusia dikembangkan dalam proyek ini untuk mendukung kemampuan-kemampuan tersebut terutama dalam berinteraksi dengan pasien dan tenaga kesehatan. Sistem deteksi menggunakan sensor jarak \lidar\ untuk mengenali lingkungan dengan pemrograman menggunakan bahasa \textit{python}. \lidar\ yang digunakan dapat menangkap informasi lingkungan hingga posisi $360\degree$ dari robot dengan sistem deteksi manusia yang dapat mendeteksi lebih dari 90\% posisi manusia dalam radius 1 m dengan benar.

Pendeteksian objek adalah salah satu teknik yang sangat berguna dan populer tetapi juga rumit dikembangkan untuk pemrosesan \textit{vision} komputer maupun robot. Sistem pendeteksi yang ada saat ini banyak yang memerlukan komputasi tinggi karena memerlukan banyak data sampel untuk melakukan pengelompokan. Pengaplikasian metode deteksi sederhana pada proyek ini dengan \lidar\ 2D memberi keuntungan menghemat komputasi dan waktu pemrosesan karena berhasil memanfaatkan informasi terbatas untuk mendeteksi manusia tanpa pengumpulan dan pemrosesan sampel. 
Sensor \lidar\ merupakan jenis sensor yang umum digunakan pada robot karena dapat digunakan pada berbagai kondisi lingkungan dengan jarak deteksi yang jauh. Proyek \textit{capstone} ini menggunakan sensor jarak yang menyajikan data dalam bentuk 2 dimensi sehingga menghemat komputasi dan memiliki harga ekonomis yang lebih murah dibanding sensor jarak canggih lainnya.  

Data mengenai posisi objek di sekitar robot dapat diterima dan diproses dengan cepat melalui sensor \lidar\ dan metode pendeteksian manusia ini. Pengelompokan objek hanya terdiri dari lingkaran dan garis karena sudah dapat mewakili berbagai bentuk objek dalam ruangan yang biasanya ada. Pada jarak sekitar 1 m dari robot, sistem deteksi ini berhasil memperoleh akurasi hingga 87\% untuk pendeteksian lingkaran dan 98\% untuk pendeteksian manusia. 
Sistem deteksi manusia ini dapat dijadikan solusi untuk pengembangan berbagai kemampuan robot seperti tracking, asistensi tenaga manusia, pelayanan, dan berbagai fungsi yang melibatkan interaksi dengan manusia hingga menjaga kenyamanan dan keamanan lingkungan.





