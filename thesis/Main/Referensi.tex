\bibitem{a1}
H. Lu, C. W. Stratton, and Y. Tang, “Outbreak of pneumonia of unknown etiology in Wuhan, China: The mystery and the miracle,” Journal of Medical Virology, vol. 92, no. 4, pp. 401–402, Feb. 2020, doi: 10.1002/jmv.25678.

\bibitem{a2}
Gorbalenya A.E.A. Severe acute respiratory syndrome-related coronavirus: the species and its viruses – a statement of the Coronavirus Study Group. BioRxiv. 2020 doi: 10.1101/2020.02.07.937862.

\bibitem{a3}
Satuan Tugas Penanganan COVID-19, “Beranda $|$ Gugus Tugas Percepatan Penanganan COVID-19,” covid19.go.id, 2021. https://covid19.go.id/ (diakses pada 25 Oktober 2022).

\bibitem{a4}
WorldOMeter, “Coronavirus Toll update: Cases \& Deaths by Country,” Worldometers, 2022. https://www.worldometers.info/coronavirus/ (diakses pada 25 Oktober 2022).

\bibitem{a5}
World Health Organization, “The Impact of COVID-19 on Health and Care Workers: a Closer Look at Deaths,” www.who.int, Sep. 13, 2021. https://www.who.int/publications/i/item/WHO-HWF-WorkingPaper-2021.1 (diakses pada 11 November 2021).

\bibitem{a6}
C. Mutia Annur, “Sebanyak 2.029 Tenaga Kesehatan Meninggal Akibat Covid-19,” databoks.katadata.co.id, Sep. 21, 2021. https://databoks.katadata.co.id/datapublish/2021/09/15/sebanyak-2029-tenaga-kesehatan-meninggal-akibat-covid-19 (diakses pada 11 November 2021).

\bibitem{a7}
L. Lacina, “How to Protect Health Workers now: WHO COVID-19 Briefing,” World Economic Forum, Apr. 10, 2020. https://www.weforum.org/agenda/2020/04/10-april-who-briefing-health-workers-covid-19-ppe-training/ (diakses pada 25 Oktober 2022).

\bibitem{a8}  
J. Gómez Rivas, C. Toribio-Vázquez, M. Taratkin, J. L. Marenco, and R. Grossmann, “Autonomous robots: a new reality in healthcare? A project by European Association of Urology-Young Academic Urologist group,” Current Opinion in Urology, vol. 31, no. 2, pp. 155–159, Dec. 2020, doi: 10.1097/mou.0000000000000842.

\bibitem{a9} 
“Autonomous robots check patients in at Belgium hospitals,” Innovation Origins, Jul. 01, 2020. https://innovationorigins.com/en/autonomous-robots-check-patients-in-at-belgium-hospitals/ (diakses pada 24 Januari 2022).

\bibitem{a10}
A. S. CNN, “Rwanda has enlisted anti-epidemic robots in its fight against coronavirus,” CNN, May 25, 2020. https://edition.cnn.com/2020/05/25/africa/rwanda-coronavirus-robots/index.html (diakses pada 26 Oktober 2022).

%BAB 2:
\bibitem{b1}
E. Guizzo, “Types of Robots - ROBOTS: Your Guide to the World of Robotics,” robots.ieee.org, 2011. https://robots.ieee.org/learn/types-of-robots/ (diakses pada 26 Oktober 2022).

\bibitem{b2}
G. A. Bekey, Autonomous Robots, vol. 24. MIT Press, 2005.

\bibitem{b3}
D. R. Parhi and B. B. V. L. Deepak, “Kinematic model of three wheeled mobile robot,” Mechanical Engineering Research, vol. 3, pp. 307–318, 2011.

\bibitem{b4}
E. Ivanjko, T. Petrinic, and I. Petrović, “MODELLING OF MOBILE ROBOT DYNAMICS,” 2010.

\bibitem{b5}
Appin Knowledge Solutions, Robotics. Jones \& Bartlett Learning, 2008.

\bibitem{b6}
P. G. Allen, “RGB-D Object Recognition and Detection,” www.cs.washington.edu. https://www.cs.washington.edu/research-projects/robotics/rgbd-object-recognition-and-detection (diakses pada 11 November 2021).

\bibitem{b7}
H. Sarbolandi, D. Lefloch, and A. Kolb, “Kinect range sensing: Structured-light versus Time-of-Flight Kinect,” Computer Vision and Image Understanding, vol. 139, pp. 1–20, Oct. 2015, doi: 10.1016/j.cviu.2015.05.006.

\bibitem{b8}
S. Foix, G. Alenya and C. Torras, "Lock-in Time-of-Flight (ToF) Cameras: A Survey," in IEEE Sensors Journal, vol. 11, no. 9, pp. 1917-1926, Sept. 2011, doi: 10.1109/JSEN.2010.2101060.

\bibitem{b9}
P. L. Rosin, Y.-K. Lai, L. Shao, and Y. Liu, RGB-D Image Analysis and Processing. Springer, 2020.

\bibitem{b10}
A. Criminisi, P. Perez, and K. Toyama, “Region Filling and Object Removal by Exemplar-Based Image Inpainting,” IEEE Transactions on Image Processing, vol. 13, no. 9, pp. 1200–1212, Sep. 2004, doi: 10.1109/tip.2004.833105.

\bibitem{b11}
S. Lu, X. Ren and F. Liu, "Depth Enhancement via Low-Rank Matrix Completion," 2014 IEEE Conference on Computer Vision and Pattern Recognition, 2014, pp. 3390-3397, doi: 10.1109/CVPR.2014.433.

% masuk sensor bukan kamera
\bibitem{bs1}
A. Carullo and M. Parvis, “An ultrasonic sensor for distance measurement in automotive applications,” IEEE Sensors Journal, vol. 1, no. 2, p. 143, 2001, doi: 10.1109/jsen.2001.936931.

\bibitem{bs2}
M. Purushottam, “Ultrasonic Sensor,” novation.ai, Jun. 23, 2021. https://novation.ai/courses/introduction-to-robotics-2/ultrasonic\_sensor/ (diakses pada 22 November 2021).

\bibitem{bs3}
SparkFun Electronics, “Infrared Proximity Sensor - Sharp GP2Y0A21YK - SEN-00242 - SparkFun Electronics,” Sparkfun.com, Jan. 08, 2015. https://www.sparkfun.com/products/242 (diakses pada 10 November 2022).

\bibitem{bs4}
InfraTec, “Infrared Sensor,” www.infratec.eu, 2021. https://www.infratec.eu/sensor-division/service-support/glossary/infrared-sensor/

\bibitem{bs5}
S. A. Daud, S. S. Mohd Sobani, M. H. Ramiee, N. H. Mahmood, P. L. Leow and F. K. Che Harun, "Application of Infrared sensor for shape detection," 2013 IEEE 4th International Conference on Photonics (ICP), 2013, pp. 145-147, doi: 10.1109/ICP.2013.6687095.

\bibitem{bs6}
S. Kumpakeaw, "Twin low-cost infrared range finders for detecting obstacles using in mobile platforms," 2012 IEEE International Conference on Robotics and Biomimetics (ROBIO), 2012, pp. 1996-1999, doi: 10.1109/ROBIO.2012.6491261.

%masuk lidar
\bibitem{bs7}
G. Urumov, “3D vs. 2D Sensor Data for Machine Perception,” understand.ai, Mar. 29, 2021. https://understand.ai/blog/annotation/machine-learning/autonomous-driving/2021/03/29/3D-vs-2D-sensor-data-for-machine-perception.html (diakses pada 22 November 2021).

\bibitem{bs8}
P. Prasher, “LiDAR, Radar, or Camera? Demystifying the ADAS / AD Technology Mix,” LeddarTech, Jun. 04, 2019. https://leddartech.com/lidar-radar-camera-demystifying-adas-ad-technology-mix/ (diakses pada 12 November 2021).

\bibitem{bs9}
N. Baghdadi and Mehrez Zribi, Optical Remote Sensing of Land Surfaces : Techniques and Methods. London: Iste Press Ltd. ; Oxford, 2016, pp. 201–247.

%masuk subbab ros dan object recognition
\bibitem{br1}
L. Joseph, Robot Operating System for Absolute Beginners : Robotics Programming Made Easy. New York, Ny: Apress, 2018.

\bibitem{br2}
Marco Alexander Treiber, An Introduction to Object Recognition. London Springer London, 2010.

%bab 3
\bibitem{c1}
Á. M. Guerrero-Higueras et al., “Tracking People in a Mobile Robot from 2D LIDAR Scans Using Full Convolutional Neural Networks for Security in Cluttered Environments,” Frontiers in Neurorobotics, vol. 12, Jan. 2019, doi: 10.3389/fnbot.2018.00085.

\bibitem{c1b}
J. H. Lee, T. Tsubouchi, K. Yamamoto and S. Egawa, "People Tracking Using a Robot in Motion with Laser Range Finder," 2006 IEEE/RSJ International Conference on Intelligent Robots and Systems, 2006, pp. 2936-2942, doi: 10.1109/IROS.2006.282147.

\bibitem{c2}
Z. Zainudin, S. Kodagoda, and G. Dissanayake, “Torso Detection and Tracking using a 2D Laser Range Finder,” Jan. 2010.

\bibitem{c2b}
K. O. Arras, O. M. Mozos and W. Burgard, "Using Boosted Features for the Detection of People in 2D Range Data," Proceedings 2007 IEEE International Conference on Robotics and Automation, 2007, pp. 3402-3407, doi: 10.1109/ROBOT.2007.363998.

\bibitem{c4}
K. R. S. Kodagoda, W. S. Wijesoma, and A. P. Balasuriya, “Road curb and intersection detection using a 2D LMS,” IEEE/RSJ International Conference on Intelligent Robots and System, Dec. 2002, doi: 10.1109/irds.2002.1041355.

\bibitem{c3}
S. Sampath, “People Counting and Tracking Based on LiDAR Data,” MSc Thesis, Queen Mary, University of London, 2019.

\bibitem{d00}
L. Suyanti, “Denah Ruangan,” jteti.ugm.ac.id, Mar. 21, 2022. https://jteti.ugm.ac.id/denah-ruangan/ (diakses pada 10 November 2022).

\bibitem{d0}
T. Klaas-Witt and S. Emeis, “The five main influencing factors for lidar errors in complex terrain,” Wind Energy Science, vol. 7, no. 1, pp. 413–431, 2022, doi: 10.5194/wes-7-413-2022.

\bibitem{d3}
Gao, Haiming \& Zhang, Xuebo \& Fang, Yongchun \& Yuan, and Jing, “A Line Segment Extraction Algorithm Using Laser Data Based on Seeded Region Growing,” International Journal of Advanced Robotic Systems, vol. 15, Feb. 2018, doi: 10.1177/1729881418755245.

\bibitem{d1}
WaveMetrics, “Errors-in-Variables Fitting,” www.wavemetrics.com. https:// www.wavemetrics.com/products/igorpro/dataanalysis/curvefitting/errorsinvariables (diakses pada 13 Oktober 2022).

\bibitem{d4}
Jssuriyakumar, “Orthogonal Distance Regression Using SciPy,” GeeksforGeeks, Jan. 02, 2022. https://www.geeksforgeeks.org/orthogonal-distance-regression-using-scipy/ (diakses pada 16 Oktober 2022).

\bibitem{d2}
Elby, “Least squares circle—SciPy Cookbook documentation,” scipy-cookbook.readthedocs.io, Mar. 22, 2011.
https://scipy-cookbook.readthedocs.io/items/Least\_Squares\_Circle.html (diakses pada 13 Oktober 2022).

\bibitem{e1}
SLAMTEC, “RPLIDAR-A2 Laser Range Scanner,” www.slamtec.com. https:// www.slamtec.com/en/Lidar/A2 (diakses pada 13 Oktober 2022).

\bibitem{f1}
H. Siswoyo, “The Role of Robots in COVID-19 Era $|$ Swinburne University, Sarawak, Malaysia,” www.swinburne.edu.my, Oct. 28, 2020. https://www.swinburne.edu.my/campus-beyond/role-robots-covid-19-era.php (diakses pada 11 November 2022).

\bibitem{f2}
R. Murphy, “Robots Have Demonstrated Their Crucial Role in Pandemics - and How They Can Help for Years to Come,” World Economic Forum, May 06, 2020. https://www.weforum.org/agenda/2020/05/robots-coronavirus-crisis/ (diakses pada 11 November 2022).

\bibitem{f3}
S. A. Bagloee, M. Tavana, M. Asadi, and T. Oliver, “Autonomous vehicles: challenges, opportunities, and Future Implications for Transportation Policies,” Journal of Modern Transportation, vol. 24, no. 4, pp. 284–303, Aug. 2016, doi: 10.1007/s40534-016-0117-3.