Wabah \covid\ telah menyebar sejak tahun 2019 dan telah menyebabkan banyak korban jiwa karena tingkat penyebaran tinggi. Tenaga kesehatan merupakan salah satu pihak yang selalu berhubungan langsung dengan pasien sehingga memiliki kemungkinan besar dalam hal tertular. Sedikitnya jumlah tenaga kesehatan dibanding pasien menyebabkan kelelahan dan stres kerja. Salah satu solusi masalah ini adalah pengembangan robot \covid\ untuk membantu tenaga kesehatan dengan sistem pendeteksi manusia untuk memudahkan interaksi robot dengan lingkungan sekitarnya. 

Sistem deteksi manusia proyek \textit{capstone} ini menggunakan metode \textit{hip detection}.
Sistem dikembangkan untuk mengolah data \lidar\ 2D yang akan membedakan manusia dengan benda di sekitarnya. Data mentah akan diklasifikasi menjadi garis dan lingkaran kemudian lingkaran yang menyerupai kaki dianggap sebagai manusia. Metode \textit{orthogonal distance regression} dan \textit{least-square} sebagai metode \textit{fitting} segmen dan deteksi dilakukan dengan algoritma \textit{seed-segment growing}. 

Data hasil pembacaan \lidar\ diidentifikasi yang memenuhi syarat sebagai kelompok segmen untuk dideteksi bentuknya. Objek lingkaran yang terkumpul kemudian dianalisis termasuk kaki manusia atau tidak. Akurasi yang diperoleh untuk pendeteksian manusia dan lingkaran dapat mencapai nilai lebih dari $80\%$ pada sistem ini. Hasil pembacaan data dan sistem deteksi ditampilkan bersamaan dengan simulasi robot berjalan di sekitar ruangan.
