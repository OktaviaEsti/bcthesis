% Pada bab ini, mahasiswa melakukan analisis mengenai dampak dari solusi yang diajukan dalam konteks global, ekonomis, lingkungan, dan sosial. Anda dapat melakukan prediksi atau analisis apa yang akan terjadi kalau implementasi  diadopsi secara luas.

Pandemi \covid\ telah memengaruhi kehidupan manusia dalam segala hal. Rutinitas harian telah berubah dan semua orang berusaha beradaptasi dengan rutinitas normal baru seperti menjaga jarak fisik, memberikan tindakan pencegahan ekstra pada kebersihan pribadi, serta meminimalkan aktivitas kita yang melibatkan kontak dekat dengan orang lain. Pandemi juga mempengaruhi berbagai perkembangan yang dilakukan untuk mengurangi dampak \covid\ dan mempercepat pemulihan, salah satunya di bidang teknologi \auto. Teknologi \auto\ yang dibahas pada proyek ini adalah robot untuk membantu tenaga kesehatan. 

Negara-negara lain telah mengembangkan berbagai robot \auto\ sebagai bantuan tenaga untuk rumah sakit. UVD Robot adalah sebuah perusahaan robotika yang berbasis di Denmark yang telah mengembangkan dan memproduksi berbagai robot desinfeksi ultraviolet (UV) \auto\ yang mampu mendesinfeksi tempat-tempat seperti rumah sakit, bandara, sekolah, hotel, dan fasilitas umum lainnya\cite{f1}.
% https://www.swinburne.edu.my/campus-beyond/role-robots-covid-19-era.php
Perusahaan robotika Malaysia DF Automation mengembangkan robot pengiriman dengan nama Makcik Kiah 19 atau MCK19 untuk membantu rumah sakit dalam mengirimkan makanan dan obat-obatan kepada pasien \covid\cite{f2}.
% https://www.weforum.org/agenda/2020/05/robots-coronavirus-crisis/
Robot ini membantu mengurangi beban kerja dan menghilangkan risiko petugas kesehatan terinfeksi oleh pasien positif \covid. Penggunaan robot sebagai pengganti manusia juga berarti bahwa akan ada lebih sedikit APD (Alat Pelindung Diri) yang digunakan yang tentunya membantu dalam mengatasi kekurangan pasokan. Robot-robot tersebut membuktikan manfaat teknologi \auto\ dalam menghadapi pandemi \covid. Robot dapat memberikan pelayanan yang konsisten, dalam jangka waktu panjang, dan tanpa kelelahan.

Manfaat potensial dari penggunaan robot \auto\ dalam kondisi tertentu dapat meningkatkan keselamatan, mengurangi tenaga manusia, mengurangi biaya, peningkatan aksesibilitas, dan penghematan waktu. Efek negatifnya bisa berupa masalah etika ketika robot membuat keputusan yang salah sehingga terjadi kecelakaan yang tidak dapat dihindari, sehingga robot perlu dikembangkan dengan teliti agar meningkatkan keselamatan penggunanya.
Operasi robot \auto\ terdiri dari tiga langkah berturut-turut: \textit{"sense"}, \textit{"plan"}, dan \textit{"act"} yang meliputi pemantauan lingkungan untuk mengumpulkan informasi, membuat keputusan berdasarkan informasi, dan bertindak sesuai keputusan\cite{f3}.
Pada langkah "pengindraan", perangkat pengindraan memperoleh informasi dari lingkungan sekitarnya yang meliputi berbagai jenis termasuk radar dari berbagai rentang, \lidar, kamera, sensor ultrasonik dan inframerah, dan Sistem Pemosisian Geografis (GPS). Perangkat yang digunakan dalam proyek ini adalah \lidar\ karena keunggulannya . 

\lidar\ 2D merupakan salah satu jenis \lidar\ yang banyak digunakan pada robot. \lidar\ 2D memiliki keuntungan harga yang lebih ekonomis dan performa yang lebih cepat dibanding jenis \lidar\ lain yang banyak digunakan pada robot. \lidar\ juga memiliki keuntungan untuk dapat digunakan dalam berbagai kondisi lingkungan pada siang dan malam.
Pengembangan pengolahan data yang berbasis \lidar\ 2D akan memudahkan pengembang robot untuk meningkatkan berbagai kemampuan robot yang melibatkan pendeteksian lingkungan yang dalam proyek \capstone\ ini adalah pendeteksian manusia. Pemanfaatan sumber data \lidar\ 2D menjadi sistem pendeteksi manusia dapat menghemat kebutuhan sensor visualisasi tambahan dan menghemat waktu pemrosesan. Pendeteksian manusia merupakan kemampuan yang dapat diaplikasikan pada berbagai jenis robot termasuk robot pelayanan kesehatan. Informasi yang disajikan dalam luaran sistem deteksi ini dapat dikembangkan untuk menambah kemampuan robot dalam berbagai fungsi untuk berinteraksi dengan manusia. 

Metode pendeteksian objek mempunyai aplikasi yang luas dalam berbagai area seperti dalam robotika, analisa gambar dalam kesehatan, pengamatan, dan interaksi pada komputer. Metode yang banyak dikembangkan saat ini mampu bekerja dengan baik namun memerlukan banyak sampel untuk melakukan pelatihan hingga memperoleh hasil yang bagus. Terdapat juga metode berbasis pendeteksian fitur yang mengestimasi bentuk 3D target yang memerlukan komputasi tinggi. Metode yang diusulkan pada \capstone\ ini memberikan solusi yang menangani kekurangan-kekurangan tersebut. Pendeteksian manusia dengan metode berbasis fitur yang tidak memerlukan \textit{training sample} dan berbasis data 2D mampu melakukan deteksi dengan biaya komputasi yang rendah dan hemat waktu.













% Autonomous vehicle operation consists of three consecutive steps: “sense,” “plan,” and “act” which include monitoring the environment for collecting information, making decision based on the information, and acting accordingly (Bagloee et al. 2016). 








% The use of this robot greatly reduces the risk of laboratory personnel from being exposed to the live virus as well as to minimise the risk of accidents and mishaps during the testing process. Besides that, the robot is proven to increase the speed and capacity of testing process as compared to manual handling by human.

